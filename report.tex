
\documentclass[10pt]{article}
\title{\textbf{Machine Learning Project Report }}
\author{Michele Orr\`u}
\usepackage{graphicx}
\usepackage{hyperref}
\usepackage{mathtools}
\begin{document}
\maketitle

\begin{abstract}
The purpose of this project is to implement a two-step analysis and classification of magnetoencephalography (MEG) data. 
\end{abstract}



\section{Introduction}
Given a user shifting his attention to the left or to the right and monitoring the spatial and temporal status of the MEG signal, the classificator should be able to \textit{a-posteriori} brain-decode towards which direction the user was actually shifting his attention.



\section{Dataset}
The phenomenom can be described as a tuple $(trial, direction)$, where:
$$
trial  = 
 \begin{bmatrix}
   freq_{0, 0} & freq_{0, 1} & \cdots & freq_{0, t}  \\
   freq_{1, 0} & freq_{1, 1} & \cdots & freq_{1, t}  \\
   \vdots      & \vdots      & \ddots & \vdots       \\
   frew_{n, 0} & freq_{n, 1} & \cdots & freq_{n, t} 
 \end{bmatrix}
\hspace{30pt}
 direction = \left\{
 \begin{array}{l l}
   -1 & \text{if left}\\
   +1 & \text{if right}\\
 \end{array}\right .
$$
$freq$ is the MEG frequency, sonded for $t$ uniform intervals of time on each of the $n = 127$ channels.
Totally, we have $< 200$ monitored phenomenons.

\vspace{10in}
\textbf{TODO}:
\begin{enumerate}
\item come \`e fatto $freq$?
\item quanti sono gli istanti di tempo $t$?
\item quanti sono i fenomeni monitorati, esattamente?
\item describe how to parse data and how to organize for testing
\end{enumerate}




\section{Problem}
Basically the task consists in producing a binary classifier ($-1$ left, $+1$ right), though its implementation has been approached chaining two different classifiers by pipeline:
\begin{enumerate}
\item the first classifier filters the data for a specific channel and attemps to give a specific classification (left, right); 
\item the second one uses as input the various classifications on all channels for a specific trial, and expresses the result still in a binary classification left/right.
\end{enumerate}



\section{Implementation}
The implementation is done in pure python, using \href{http://scikit-learn.org/stable/}{scikit-learn} module. 

\vspace{10in}
\textbf{TODO}:
\begin{enumerate}
\item  \href{https://en.wikibooks.org/wiki/LaTeX/Algorithms_and_Pseudocode}{pseudocodice per pseudoscienziati}
\item introduction to the source code structure
\item documentation
\end{enumerate}

s

\section{Conclusions}
\textbf{TODO}



\end{document}

