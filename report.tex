\documentclass[10pt]{article}
\title{\textbf{Machine Learning Project Report }}
\author{Michele Orr\`u}
\usepackage{graphicx}
\usepackage{hyperref}
\usepackage{mathtools}
\begin{document}
\maketitle

\begin{abstract}
The purpose of this project is to implement a two-step analysis and classification of magnetoencephalography (MEG) data. 
\end{abstract}



\section{Introduction}
Given a user shifting his attention to the left or to the right and monitoring the spatial and temporal status of the MEG\cite{Biomag2010} signal, the goal is to build a classifier capable of brain-decode \textit{a-posteriori} towards which direction the user was actually shifting his attention.



\section{Dataset}
The phenomenom can be described as a set $\{(trial_i, direction_i)\}_{i= 1, 2, \dots, trials}$, where:
$$
trial  = 
 \begin{bmatrix}
   freq_{0, 0} & freq_{0, 1} & \cdots & freq_{0, t}  \\
   freq_{1, 0} & freq_{1, 1} & \cdots & freq_{1, t}  \\
   \vdots      & \vdots      & \ddots & \vdots       \\
   frew_{n, 0} & freq_{n, 1} & \cdots & freq_{n, t} 
 \end{bmatrix}
\hspace{30pt}
 direction = \left\{
 \begin{array}{l l}
   -1 & \text{if left}\\
   +1 & \text{if right}\\
 \end{array}\right .
$$
$freq$ is the MEG frequency, sonded for $t$ bands on each of the $n$ channels.

More precisely, $freq$ encodes the mean $t=6$ different bands:
$2-4Hz, 4-8Hz, 8-13Hz, 13-20Hz, 20-35Hz, 35-46Hz$ taken from a single ($1$) subject on $n=274$ channels. 
Globally, we have $255$ trials ($127$ left, $128$ right).  

Input data is a precious resource, and given that at least $10\%$ is going to be used for testing, 

\subsection{TODO}
\begin{enumerate}
\item quanti sono gli istanti di tempo $t$?
\item \`e veramente la media delle varie frequenze quella che entra in gioco? 
\item la percentuale di dati per fare il test \`e veramente il dieci per cento?
\item quanti sono i fenomeni monitorati, esattamente?
\item describe how to parse data and how to organize for testing
\end{enumerate}




\section{Problem}
Though basically the task consists in a binary classifier ($-1$ left, $+1$ right), the problem has been approached chaining by pipeline two different learning models:
\begin{enumerate}
\item the first one, filters the data for a specific channel and attemps to give a specific classification (left, right); 
\item the second one, uses as input the various classifications on all channels for a specific trial, and expresses the result still in a binary classification left/right.
\end{enumerate}


\subsection{Channel Classification}

The first classifier is a vector of functions
$ \{f_0, f_1, \dots, f_n \}$ 
, one for each channel, defined:
$ f: (\begin{array}{l l l} freq_{i, j}, & \cdots, & freq_{i, j} \end{array}) \to [-1, 1]  $
 expressing the best direction ($left=-1$, $right=+1$), in terms of \textit{most probable}. Since the dataset is known, but the desidered output is really difficult to argue and test, the learning model with easiest implementation is probably \textbf{Bayes}. In fact, we know the mean $\mu$ and we can play around with the covariance matrix $\Sigma$



\subsection{Final Classification}
The second classifier's task is to find a function
 $$
g: [-1, +1]^n \to \{-1, +1\}
$$
    ,  expressing the direction chosen by the user. In this case instead, we have desidered output known but a really variable input space. Hence, $g$ is a non-probabilistic, binary linear classifier, perfectly fitting the definition of a \textbf{SVM}.

\subsection{TODO}
\begin{enumerate}
\item quali sono le motivazioni per aver concatenato due classificatiori? se non ricordo male era lo studio delle aree del cervello. giusto?
\item would be ok, second classifier? 
\item is rally the best, first classifier?
\end{enumerate}

\section{Implementation}
The implementation is done in pure python, using \href{http://scikit-learn.org/stable/}{scikit-learn} module. 

\subsection{TODO}:
\begin{enumerate}
\item  \href{https://en.wikibooks.org/wiki/LaTeX/Algorithms_and_Pseudocode}{pseudocodice per pseudoscienziati}
\item introduction to the source code structure
\item documentation
\end{enumerate}



\section{Conclusions}
Machine Learning rocks.

\begin{thebibliography}{9}

\bibitem{Biomag2010}
Marcel van Gerven, Ole Jensen,
\emph{Attention modulations of posterior alpha as a control signal for two-dimensional brain–computer interfaces},
Journal of Neuroscience Methods, 2009
\end{thebibliography}

\end{document}

