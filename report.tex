\documentclass[10pt]{article}
\title{\textbf{Machine Learning Project Report }}
\author{Michele Orr\`u}
\usepackage{graphicx}
\usepackage{mathtools}
\usepackage{amsfonts}
\usepackage{unitsdef}
\usepackage{hyperref}
\begin{document}
\newcommand{\code}[1]{\texttt{#1}}
\maketitle
\begin{abstract}
The purpose of this project is to implement a two-step analysis and classification of magnetoencephalography(MEG) data. 
\end{abstract}


\section{Introduction}
Given a user shifting his attention to the left or to the right and monitoring the spatial and temporal status of the MEG\cite{Biomag2010} signal, the goal is to build a classifier capable of brain-decode \textit{a-posteriori} towards which direction the user was actually shifting his attention.



\section{Dataset}
\label{dataset}
in the following we describe how the input data was given, and how it is structured. 

\noindent
Input data was taken from the Biomag 2010 contest, first analyzed in 
\cite{Biomag2010} and then in \cite{Braindecoding}: it consists in
 magnetoencephalography(MEG) data collected from 15 subjects who were engaged in
a task where they covertly had to visually attend left, right, up, or down during a
period of 2500\milli\second. 
Then, MEG data has been filtered for left, right directions, transformated using spectral energy distribution and compressed in two numpy pickled binary files
 (\code{.npz}), namely \code{freq\_left\_gradient.npy.npz}
 and \code{freq\_left\_gradient.npy.npz}.

More formally, those files contains the dump of a dict-like area of memory, containing keys:
\begin{verbatim}
>>> np.load('freq_left_gradient.npy.npz')
<numpy.lib.npyio.NpzFile object at 0x251cad0>
>>> print _.keys()
['sample_freq', 'lf_freq', 'time', 'label', 'fsample']
\end{verbatim}

\code{time}, \code{label}, and \code{sample\_freq} are vectors labelling respectively when, where, and on which frequency data was sampled.
\code{lf\_freq} contains the real inputs (N.B., right samples contains key \code{rg\_freq}), espressed as a matrix $(127, 274, 376)$ $trials \times channel \times frequency$.

\paragraph{Preprocessing} is implemented in the \code{preprocess()} function, in order to reduce dimensionality of the input space, and have quicker access to the features. It simply extracts data from the previously discussed files, and outputs a \code{dataset.npz} dict-like, having keys:
\begin{verbatim}
>>> np.load('dataset.npz')
<numpy.lib.npyio.NpzFile object at 0x25e31d0>
>>> _.keys()
['label', 'dataset']
\end{verbatim}
where \code{label} contains labelling information for each channel, and \code{dataset} is a vector $\{(trial_i, direction_i)\}_{i= 1, 2, \dots, trials}$, such that:

$$
trial_i  = 
 \begin{bmatrix}
   freq_{0, 0} & freq_{0, 1} & \cdots & freq_{0, b}  \\
   freq_{1, 0} & freq_{1, 1} & \cdots & freq_{1, b}  \\
   \vdots      & \vdots      & \ddots & \vdots       \\
   freq_{n, 0} & freq_{n, 1} & \cdots & freq_{n, b} 
 \end{bmatrix}
\hspace{30pt}
 direction_i = \left\{
 \begin{array}{l l}
   -1 & \text{if left}\\
   +1 & \text{if right}\\
 \end{array}\right .
$$
More precisely, $trial_i \in \mathbb{R} ^{n+b}$ represents over each row one of the $n=274$ channels sonded, and over each column the mean for each of the $b=7$ different band ranges, respectively:
$$
\{0.0 \dots 24.7,\ 24.7 \dots 49.5,\ 49.5 \dots 74.2,\ 74.2 \dots 98.9,\ 98.9 \dots 123.7,\ 123.7 \dots 148.4\} \ \hertz
$$
Globally, we have $255$ trials ($127$ left, $128$ right).


\section{Problem Formulation}
The problem of classifying the shift of attention to
the left ($-1$) or to the right ($-1$) has been
approached by stacking\cite{Wolpert92} two different
models. The first one consists in a vector of
learning functions, one for each channel, expected to
predict the \emph{local} direction; the second one
instead, is expected to predict the \emph{global}
output, given the local information.

Reasons for this choices are to be found mostly in a reserach interest concerning localized informations of the brain.
Furthermore, by stacking, we should presumibly
reduce the dimensionality of the feature space, and reach a better balance between the number of instances and the number of features.

\subsection{Formalization} 

\paragraph{First Step} as stated above, consists in a vector of functions
$ \{f_0, f_1, \dots, f_n \}$ 
taking as input the frequency bands over a specific channel and expressing the best direction ($left$, $right$), in terms numeric values $(-1, +1)$. Hence, we can define them as:
$$
\begin{array}{lr}
f_k: (\begin{array}{l l l} freq_{k, 0}, & \cdots, & freq_{k, t} \end{array})^{trials} \to \{-1, 1\}^{trials},
&
k \in \{0, \dots, n\}
\end{array}
$$ 

\paragraph{Second Step} we are looking for a funciton expressing the global output as a boolean value, given the localized information about each channel previously discovered and then aggregated:
$$
g: \{-1, +1\}^n \to \{-1, +1\}
$$


\subsection{Learning Model}
\label{learning_model}

\paragraph{First Step} each of the ${f_k}$ function has been trained using a \textbf{non-linear SVM classifier}, with \emph{gaussian} and \emph{polynomial} kernels. 

\paragraph{Second Step} $g$ instead, was trained using a simple \textbf{norm-one classifier}
Norm-one classifiers simpl
e attempt to define the linear hyperplane 
$g(\boldsymbol{x}) = w^T\boldsymbol{x} + w_0$ using $||w||_1$


\subsection{Evaluations for Stacking}

The stacked architecture adds extra complexity to the model for feature selection. In fact, constraints for features selection must be that: 
\begin{itemize}
\item $f_k$'s training set \emph{must} not have examples in common with the validation set;
\item $g$'s training set \emph{must} not have examples in common with the validation set;
\item $f_k$'s training set \emph{should} have null intersection with the $g$ training set.
\end{itemize}

Hence, for supponting the the data in training and validation set; and another one inner, using the training data of the outer set for training both the $f_k$ and $g$. 

\subsection{Training}

The first step is to train each channel-specific learner, using the selected features for training, filtered for the specic channel. 
\begin{verbatim}
fst[channel].fit(input[:, channel], output)
\end{verbatim}

\noindent
Then, the second classifier is trained using the selected features for the second training step, applied to the first layer of learners.
\begin{verbatim}
input = np.array([fst[channel].predict(input[:, channel])
                   for channel in xrange(channels)]).T
snd.fit(input, output)

\end{verbatim}


\subsection{Validation}
Each round of cross-validation involves a combination of $f_k$s and $g$ which is eventually saved to disk if its accuracy is over $75\%$. 
Globally, only the tuple of learners having best accuracy is considered.
Besides all this, the best learnerns achieved by constraints tuning has been selected by examining the output logs \code{output.log}, \code{output.log.1}, \code{output.log.2}.

\section{Implementation}
The implementation is done in pure 
\href{https://python.org/}{python}
, using the
\href{http://scikit-learn.org/stable/}{scikit-learn}
module. Documentatios has been automatically generated by
\href{http://sphinx-doc.org/}{Sphinx}, the most widely used documentation generator in pyton.

Source code is available on 
\href{https://github.com/mmaker/braindecoding}{GitHub}.

\subsection{Code Structure}
File \code{braindecoding.py} handles preprocessing, learning and validation. It is composed of three main functions, each one bindend to a specific command: 
\begin{itemize}
\item \code{preprocess()}, binded to \code{-p}, performs the preprocessing discussed in section \nameref{dataset};
\item \code{learn(parameters=None)}, binded to \code{-l}, performs the learning as described in section \nameref{learning_model};
\item \code{tune(processes=None)}, binded to \code{-t}, starts up a pool of processes, each one computing a \code{learn()} function. Used for validation.
\end{itemize}

\section{Conclusions}
Performance has been evaluated only computing the accuracy of the outer classifier. Even thouth succint, is a valuable method for having an idea of the flexibility of the learner.

Examining files \code{output.log*} shown that \emph{polynomial} kernels had a higher density of learners above the $75\%$ of accuracy.

\paragraph{Further improvements} of the learner could have been achieved defining the codomain of each $f_k$ as an interval $[-1, +1]$, using \emph{support vector regression}'s implementation \code{sklearn.svm.SVR}, which has not been done for a matter of time.
Another side on which the learner is currently inefficient, is the lack of information about the \emph{spread} of frequencies, once resized. Adding the standard deviation, next to the mean of the frequencies, as inputs for the first layer of learners should fix that; but still, they have not been implemented for a pure matter of time.



\begin{thebibliography}{9}

\bibitem{Biomag2010}
Marcel van Gerven, Ole Jensen,
\emph{Attention modulations of posterior alpha as a control signal for two-dimensional brain-computer interfaces},
Journal of Neuroscience Methods, 2009


\bibitem{Wolpert92}
David H. Wolpert,
\emph{Stacked Generalization},
Neural Networks, 1992

\bibitem{Braindecoding}
E. Olivetti, A. Mognon, S. Greiner and P. Avesani, 
\emph{Brain Decoding: Biases in Error Estimation}, 
Brain Decoding Workshop, 2010

\end{thebibliography}

\end{document}